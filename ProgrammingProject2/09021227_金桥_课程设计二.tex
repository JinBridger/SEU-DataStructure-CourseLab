\documentclass[a4paper]{ctexart}
\usepackage[utf8]{inputenc}
\usepackage{graphicx}
\usepackage{setspace}
\usepackage{amsmath}
\usepackage{subfigure}
\usepackage{float}
\usepackage{listings}
\usepackage{multirow}
\usepackage{ctable}
\usepackage{caption}
\usepackage{pifont}
\usepackage{enumitem}
\usepackage{enumerate}
\usepackage{booktabs}
\usepackage[colorlinks,linkcolor=red]{hyperref}
\usepackage{xcolor}
\usepackage[margin=1in]{geometry} 
\usepackage{array} 
%\renewcommand\arraystretch{1.4} 
\lstset{
    numbers=left,
    keywordstyle= \color{ blue!70},
    commentstyle= \color{red!50!green!50!blue!50},
    rulesepcolor= \color{ red!20!green!20!blue!20} ,
    escapeinside=``,
    numberstyle=\tt,
    numbersep=0em,
    xleftmargin=2em,
    breaklines,
    aboveskip=1em,
    framexleftmargin=2em,
    frame=shadowbox,
    basicstyle=\tt,
    language=C++
}
\title{数据结构课程设计二实验报告}
\author{学号:\underline{09021227}\ \ 姓名:\underline{金桥}\ \ 实验类型:\underline{必做}}
\begin{document}
\maketitle

\subsection*{1. 问题描述:描述实验内容和要求以及需要解决的问题。\\ \kaishu{提示:结合教师课堂讲授内容,仔细分析实验要求,对实验内容进行需求分析。}}

\noindent \textbf{问题描述:}

· 在ZIP压缩文档中,保留着所有压缩文件和目录的相对路径和名称。

· 当使用WinZIP等软件打开ZIP压缩文档时,可以从这些信息中重建目录的树状结构。

\noindent \textbf{实验要求:}

· 实现目录的树状结构的重建。

· $N$:取值 $1 \sim 100$ ,表示ZIP压缩文档中的文件和目录数量。

\noindent \textbf{输入文件格式:}

第一行:$N$ : 文件和目录的数量

第二行:文件或目录的相对路径和名称,每行不超过260个字符

\noindent \textbf{输出文件格式:\small{(假设:所有路径都是相对于\texttt{root}目录,每级目录比上一级目录多缩进2个空格)}}

从\texttt{root}开始输出每个目录名称

\hspace*{2em} 按字典序输出所有子目录

\hspace*{2em} \hspace*{2em} 按字典序输出所有文件

\noindent \textbf{注意:}

· 文件名称与目录名称,可能存在重名,必须进行区分。

· 路径名称与文件名称,仅包括英文字母,且字母区分大小写。

· 符号‘\textbackslash’仅作为路径分隔符,且目录以‘\textbackslash’结束。

· 不存在重复的输入值。

· 每个文件名后一对“()”内的值,为该文件的大小。

\subsection*{2. 算法思想:详细描述解决相应问题所需要的算法设计思想。}

采用\textbf{树状数据结构}实现,目录作为父节点,目录下的文件夹以及文件作为子节点。

单个节点存放节点名称,结点类型(文件夹/文件),以及子节点指针。

添加节点时,依此访问其父节点,若其父节点不存在则创建其父节点。创建叶子节点时注意判别是否重复。

在打印之前对所有节点的子节点指针序列进行字典序排序。最后进行输出。

树内部的函数大部分采用递归方式实现。

除此以外,还需要自行实现 \texttt{std::vector std::string std::sort} 的功能,前两者为\textbf{顺序表},后者为\textbf{冒泡排序}。

\newpage
\subsection*{3. 功能函数:描述所设计的功能函数。如果有多个函数,需要描述它们之间的关系。}

\noindent {\large\textbf{\texttt{MyVec} 类:}}

\noindent \textbf{\texttt{MyVec} :默认构造函数,生成一个空的 \texttt{MyVec}}.

\begin{lstlisting}
    /**
    *  @brief MyVec object constructor
    */
   MyVec();
\end{lstlisting}

\vspace{3em}

\noindent \textbf{\texttt{MyVec} :给定另一个数组的开始位置与结束位置生成对应的 \texttt{MyVec}}

通过传入指向开始元素的指针\texttt{begin}与指向末尾元素的下一个位置的指针\texttt{end}构建\texttt{MyVec}.

\begin{lstlisting}
    /**
     *  @brief MyVec object constructor, copy from begin to end
     *  @param begin - begin position
     *  @param end   - end position
     */
    MyVec(DT* begin, DT* end);
\end{lstlisting}

\vspace{3em}

\noindent \textbf{\texttt{MyVec} :拷贝构造函数}

\begin{lstlisting}
    /**
     *  @brief MyVec object copy constructor
     *  @param vec - copy vector
     */
    MyVec(const MyVec<DT>& vec);
\end{lstlisting}

\vspace{3em}

\noindent \textbf{\texttt{\textasciitilde MyVec} :析构函数}

\begin{lstlisting}
    /**
     *  @brief MyVec object destructor
     */
    ~MyVec();
\end{lstlisting}

\vspace{3em}

\noindent \textbf{\texttt{operator[]} :获取下标 $n$ 的元素}

\begin{lstlisting}
    DT& operator[](int n);
\end{lstlisting}

\vspace{3em}

\noindent \textbf{\texttt{pushBack} :向尾部插入新元素}

\begin{lstlisting}
    /**
     *  @brief push element to the end of vector
     *  @param element - data to be pushed
     */
    void pushBack(const DT& element);
\end{lstlisting}

\vspace{2em}

\noindent \textbf{\texttt{sort} :按给定方式排列所有元素}

传入比较函数,按照比较函数进行排序,类似于 \texttt{std::sort} 函数的第三个参数。

\begin{lstlisting}
    /**
     *  @brief bubble sort
     *  @param fun - sort order
     */
    void sort(bool(*fun)(DT, DT));
\end{lstlisting}

\vspace{2em}

\noindent \textbf{\texttt{begin} :获取指向首个元素的指针}

\begin{lstlisting}
    /**
     *  @brief  Return iterator to beginning
     *  @retval  - iterator to beginning
     */
    DT* begin() const;
\end{lstlisting}

\vspace{2em}

\noindent \textbf{\texttt{end} :获取指向最后一个元素的下一个位置的指针}

\begin{lstlisting}
    /**
     *  @brief  Return iterator to end
     *  @retval  - iterator to end
     */
    DT* end() const;
\end{lstlisting}

\vspace{2em}

\noindent \textbf{\texttt{size} :获取\texttt{MyVec}大小}

\begin{lstlisting}
    /**
     *  @brief  Return size
     *  @retval  - size of vector
     */
    int size() const;
\end{lstlisting}

\vspace{3em}

\noindent {\large\textbf{\texttt{MyStr} 类:}}

\noindent \textbf{\texttt{MyStr} :默认构造函数,生成一个空的 \texttt{MyStr}}.

\begin{lstlisting}
    /**
    *  @brief MyStr object constructor
    */
   MyStr::MyStr();
\end{lstlisting}

\vspace{3em}

\noindent \textbf{\texttt{MyStr} :通过字符串常量构建 \texttt{MyStr}}.

\begin{lstlisting}
    /**
    *  @brief MyStr object constructor
    *  @param c - string constant. eg. "str"
    */
   MyStr::MyStr(const char* c);
\end{lstlisting}

\vspace{3em}


\noindent \textbf{\texttt{MyStr} :拷贝构造函数}.

\begin{lstlisting}
    /**
    *  @brief MyStr copy constructor
    *  @param s - copy string
    */
   MyStr::MyStr(const MyStr& s);
\end{lstlisting}

\vspace{3em}

\noindent \textbf{\texttt{\textasciitilde MyStr} :析构函数}.

\begin{lstlisting}
    /**
    *  @brief MyStr object destructor
    */
   MyStr::~MyStr()
\end{lstlisting}

\vspace{3em}

\noindent \textbf{\texttt{operator[]} :获取字符串第 $n$ 个字符}

\begin{lstlisting}
    char& MyStr::operator[](int n);
\end{lstlisting}

\newpage

\noindent \textbf{\texttt{substr} :获取字符串 \texttt{[start, start+offset)} 位置的字串}

类似于 \texttt{std::string} 的 \texttt{substr} 函数。 例如 "abcde" 的 \texttt{substr(0, 3)} 就是 "abc"

\begin{lstlisting}
    /**
    *  @brief  get substring from [start, start + offset)
    *  @param  start  - start position
    *  @param  offset - substring length
    *  @retval        - substring
    */
   MyStr MyStr::substr(int start, int offset) const;
\end{lstlisting}

\vspace{3em}

\noindent \textbf{\texttt{length} :获取字符串长度}

\begin{lstlisting}
    /**
    *  @brief  get length
    *  @retval int - length
    */
   int MyStr::length() const
\end{lstlisting}

\vspace{3em}

\noindent \textbf{\texttt{operators} :重载的运算符函数,与 \texttt{std::string} 的运算符行为一致}

\begin{lstlisting}
    friend std::istream& operator>>(std::istream&, MyStr&);
    friend std::ostream& operator<<(std::ostream&, const MyStr&);
    friend MyStr operator+(const MyStr&, const MyStr&);
    friend bool operator==(const MyStr&, const MyStr&);
    friend bool operator<(const MyStr&, const MyStr&);
    friend bool operator>(const MyStr&, const MyStr&);
\end{lstlisting}

\vspace{3em}

\noindent {\large\textbf{\texttt{FileTreeNode} 类:}}

\noindent \textbf{\texttt{FileTreeNode} :构造函数,给定节点名称与是否为文件夹}.

\begin{lstlisting}
    /**
    *  @brief FileTreeNode object constructor
    *  @param name     - node name
    *  @param isFolder - true: node is a folder
    */
   FileTreeNode::FileTreeNode(MyStr name, bool isFolder);
\end{lstlisting}

\vspace{3em}

\noindent \textbf{\texttt{\textasciitilde FileTreeNode} :析构函数}.

\begin{lstlisting}
    /**
     *  @brief FileTreeNode object destructor
     */
    FileTreeNode::~FileTreeNode();
\end{lstlisting}

\vspace{2.5em}

\noindent \textbf{\texttt{getName} :获取节点名称}.

\begin{lstlisting}
    /**
    *  @brief  get node's name
    *  @retval  - node's name
    */
   MyStr FileTreeNode::getName() const;
\end{lstlisting}

\vspace{2.5em}

\noindent {\large\textbf{\texttt{FileTree} 类:}}

\noindent \textbf{\texttt{FileTree} :构造函数,构造一颗空的 \texttt{FileTree} 树}.

\begin{lstlisting}
    /**
    *  @brief FileTree object constructor
    */
   FileTree::FileTree();
\end{lstlisting}

\vspace{2.5em}

\noindent \textbf{\texttt{\textasciitilde FileTree} :析构函数}.

通过调用 \texttt{destructRecursion} 递归析构整棵树。

\begin{lstlisting}
    /**
    *  @brief FileTree object destructor
    */
   FileTree::~FileTree();
\end{lstlisting}

\vspace{2.5em}

\noindent \textbf{\texttt{destructRecursion} :递归析构函数}.

传入当前要析构的子树根节点\texttt{node}进行析构。

\begin{lstlisting}
    /**
    *  @brief destruct recursion
    *  @param node - current destructing node
    */
   void FileTree::destructRecursion(FileTreeNode* node);
\end{lstlisting}

\vspace{3em}

\noindent \textbf{\texttt{add} :通过 \texttt{MyVec} 添加节点}.

将字符串分解为父节点列表\texttt{prefix},节点名称\texttt{name},是否为文件夹\texttt{isFolder},传入\texttt{addNode}函数进行添加。

\begin{lstlisting}
    /**
    *  @brief add node to the tree
    *  @param pth - the path of file/folder
    */
   void FileTree::add(MyStr pth);
\end{lstlisting}

\vspace{2em}

\noindent \textbf{\texttt{addNode} :接收上一个函数传来的参数,并添加节点}.

添加节点的入口函数,调用\texttt{addNodeRecursion}完成节点的添加。

\begin{lstlisting}
    /**
    *  @brief add node to the tree
    *  @param prefix   - prefix of path
    *  @param name     - name of file/folder
    *  @param isFolder - true: node is a folder
    */
   void FileTree::addNode(MyVec<MyStr> prefix, MyStr name, bool isFolder);
\end{lstlisting}

\vspace{2em}

\noindent \textbf{\texttt{addNodeRecursion} :递归添加节点函数}.

\begin{lstlisting}
    /**
    *  @brief add node recursion
    *  @param node     - current node
    *  @param num      - current pos in prefix
    *  @param prefix   - prefix of the path
    *  @param name     - node name
    *  @param isFolder - true: node is a folder
    */
   void FileTree::addNodeRecursion(FileTreeNode* node, int num, MyVec<MyStr> prefix, MyStr name, bool isFolder);
\end{lstlisting}

\vspace{2em}

\noindent \textbf{\texttt{sortNode} :对子节点进行排序的入口函数}.

\begin{lstlisting}
    /**
    *  @brief sort all node by dict order
    */
   void FileTree::sortNode();
\end{lstlisting}

\vspace{3em}

\noindent \textbf{\texttt{sortNodeRecursion} :递归的对每个节点的子节点进行排序}.

\begin{lstlisting}
    /**
    *  @brief sort all node recursion
    *  @param node - current node
    */
   void FileTree::sortNodeRecursion(FileTreeNode* node);
\end{lstlisting}

\vspace{3em}

\noindent \textbf{\texttt{print} :打印树的入口函数}.

\begin{lstlisting}
    /**
    *  @brief print the tree
    */
   void FileTree::print() const
\end{lstlisting}

\vspace{3em}

\noindent \textbf{\texttt{printRecursion} :递归的对每个节点进行打印}.

\begin{lstlisting}
    /**
    *  @brief print recursion
    *  @param prefix - the output prefix
    *  @param node   - current node
    *  @param isLast - true: node is the last subnode
    */
   void FileTree::printRecursion(MyStr prefix, FileTreeNode* node, bool isLast) const
\end{lstlisting}

\vspace{3em}

\noindent \textbf{\texttt{getOutput} :获取输出的字符串}.

如果节点是文件夹则输出时加上下划线。
\textbf{如果出现乱码请注释掉里面的 \texttt{if} 语句}

\begin{lstlisting}
    /**
    *  @brief  get output string of name, if is a folder then give it an underline
    *  @param  name     - name
    *  @param  isFolder - true: node is a folder
    *  @retval          - output string
    */
   MyStr FileTree::getOutput(MyStr name, bool isFolder) const;
\end{lstlisting}

\vspace{3em}

\subsection*{4. 测试数据:设计测试数据,或具体给出测试数据。\\ \kaishu{提示:要求测试数据能全面地测试所设计程序的功能。}}
测试数据如下:

\noindent \textbf{测试输入数据一:}
\begin{lstlisting}[numbers=none]
7
b(115)
c\
ab\cd
a\bc(798)
ab\d(161)
a\d\a(1338)
a\d\z\    
\end{lstlisting}

\vspace{3em}

\noindent \textbf{测试输入数据二:}
\begin{lstlisting}[numbers=none]
22
System\Overview(221)
Abstraction\Notions(183)
Abstraction\Notions\Definition(1248)
Abstraction\Notions\Example(252)
Exercises\
Basice(1956)
OODesign\Evolution\Generations\History\Efficient(14)
OODesign\Evolution\Generations\History\Flexible(11)
OODesign\Evolution\Generations\History\Available(9)
System\Requirements(57)
System\Analysis(233)
OODesign\Decomposition(204)
OODesign\Programming\Object(55)
OODesign\Programming\Programming(48)
OODesign\Programming\Language(52)
System\Design(258)
System\Refinement(137)
System\Verification(451)
OODesign\Evolution\Generations\Language\First(19)
OODesign\Evolution\Generations\Language\Second(20)
OODesign\Evolution\Generations\Language\Third(13)
OODesign\Evolution\Generations\Language\Fourth(24) 
\end{lstlisting}

\subsection*{5. 测试情况:给出程序的测试情况,分析运行结果,显示实验结果截图。}

实验结果截图:

\noindent \textbf{测试输出数据一:}
\begin{center}
    \includegraphics[width=0.3\textwidth]{image/result2.jpg}
\end{center}

\vspace{2em}

\noindent \textbf{测试输出数据二:}
\begin{center}
    \includegraphics[width=0.55\textwidth]{image/result.jpg}
\end{center}

如图,文件夹带有下划线,文件无下划线。与输入内容预期结果一致。
\textbf{如果出现乱码请注释掉\texttt{FileTree::getOutput}函数里面的 \texttt{if} 语句块!}

\subsection*{6. 实验总结:写出实验过程中遇到的问题,以及问题的解决过程。分析算法的时间复杂度和空间复杂度,总结实验心得体会。}

\noindent \textbf{问题及解决过程:}

实验过程无问题。

\noindent \textbf{算法的时间复杂度分析:}

\texttt{MyVec} 设元素个数为 $n$,则单次添加为 $O(n)$,访问为$O(1)$.

\texttt{MyStr} 设元素个数为 $n$,则 \texttt{substr} 函数复杂度为 $O(n)$.

\texttt{FileTree} 添加单个文件/文件夹:设树的大小为 $N$,则添加时最坏为 $O(N)$,此时树退化为一条链。

\texttt{FileTree} 按字典序排序所有文件/文件夹:设树的大小为 $N$, 使用冒泡排序进行排序,对于一个有 $m$ 子节点的排序时间复杂度为 $O(m^2)$。最坏时树退化为一条链,复杂度为 $O(N^3)$.

\texttt{FileTree} 设树的大小为 $N$,打印树为 $O(N)$.

\noindent \textbf{算法的空间复杂度分析:} 

\texttt{MyVec} 设元素个数为 $n$ 则为 $O(n)$.

\texttt{MyStr} 设字符个数为 $n$ 则为 $O(n)$.

\texttt{FileTree} 设树的大小为 $N$ 则为 $O(N)$.

\noindent \textbf{心得体会:}

感觉 \texttt{MyVec} 单次添加的性能还可以优化,类似于 \texttt{std::vector} 一样倍增容量而不是每次都重新构建。除了 \texttt{iostream} 库以外没有使用任何STL内容。

\subsection*{7. 源代码:给出项目所有源程序清单。\\ \kaishu{建议:源程序中应有充分的注释,例如注释每个函数参数的含义、函数返回值的含义、函数的功能、主要语句段的功能,等等。}}

实验源程序清单:

\hspace*{2em} 1. \texttt{main.cpp}

\hspace*{2em} 2. \texttt{FileTree.h}

\hspace*{2em} 3. \texttt{FileTree.cpp}

\hspace*{2em} 4. \texttt{MyVec.hpp}

\hspace*{2em} 5. \texttt{MyStr.h}

\hspace*{2em} 6. \texttt{MyStr.cpp}

\hspace*{2em} 7. \texttt{input.txt}
\end{document}